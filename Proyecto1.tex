\documentclass{report}

\usepackage[spanish,activeacute]{babel}
\usepackage[utf8x]{inputenc}
\usepackage{a4}
\usepackage[spanish]{layout}
\usepackage{amssymb}
\usepackage{hyperref}
\usepackage{endnotes}
\usepackage{ragged2e}
\usepackage{multirow}
\usepackage{hyphenat}

\setlength{\oddsidemargin}{25mm}
\setlength{\evensidemargin}{25mm}
\setlength{\voffset}{6mm}
\setlength{\textwidth}{330pt}
\setlength{\footskip}{6mm}
\setlength{\textheight}{562pt}
\setlength{\marginparwidth}{35pt}
\setlength{\marginparsep}{10pt}
\setlength{\headheight}{15pt}
\usepackage[T1]{fontenc}
\usepackage{eurosym}
\usepackage{url}
\usepackage{graphicx} % Para incluir imágenes
\usepackage{wrapfig}
\usepackage{caption}
\usepackage{subcaption}
\usepackage{fixltx2e} % para los subíndices
\usepackage{color}
\usepackage{lettrine} % para letras capitales de inicios
\usepackage{setspace}
\usepackage{pdflscape} % Para entorno landscape y páginas en apaisado
\usepackage{fancyhdr}
\pagestyle{fancy}
\fancyhead{} % Borramos las cabeceras por defecto
\fancyhf{} % Borramos todas las cabeceras y estilos por defecto
\fancyhead[RO,LE]{\textbf{Proyecto Genshin Impact}}
\fancyfoot{} % Borramos pies de página por defecto
\fancyfoot[LE,RO]{\thepage}
\fancyfoot[RE,LO]{\textbf{Grupo 4}}

\renewcommand{\footrulewidth}{0.4pt}
\graphicspath{ {images/} }
\renewcommand{\notesname}{Notas}

\title{
{Proyecto Base de datos I\\
Genshin Impact}\\
{\large Universidad de Ingeniería y Tecnología}\\
{\includegraphics{logo.png}
}
}

\author{
        Harold Canto\\
        Marcelo Surco\\
        Paulo Cuaresma\\
        Alberto Rincon\\
        Juan Torres  
        } 
\date{13 de junio del 2021}


\begin{document}

\maketitle

\tableofcontents

\chapter{Requisitos}
\section{Introducción}
\paragraph{El reciente videojuego, Genshin Impact, desarrollado por el estudio MiHoyo; se ha convertido en uno de los más famosos desde su lanzamiento en el año 2020. Dicho juego fue un total éxito debido a su modalidad RPG (Role-Playing Game) de mundo abierto totalmente renovado; donde el usuario puede disfrutar de una envolvente historia junto con múltiples pruebas y desafíos que promueven el progreso de los personajes. Hoy por hoy, el estudio chino de desarrolladores, busca una sólida idea o implementación para lograr una optimización y añadir una guía bastante completa con la finalidad de ayudar a los múltiples usuarios respecto a ciertos aspectos basándose, primordialmente, en consultas sobre la base de datos del propio videojuego, Genshin Impact. Por lo tanto, el presente proyecto buscará dar solución a lo planteado; donde, con la ayuda de la respectiva base de datos, se logrará implementar un modelado completo y simplificado a lo necesario para lograr el posible consultorio de ayuda o asesoramiento de jugadores. }
\paragraph{Por tal motivo, el objetivo de esta primera entrega en el presente proyecto será de lograr representar el actual modelado del videojuego enfocado en la problemática anteriormente mencionada. Y para lograrlo, debemos construir las diversas entidades y factores que influyen y están presentes en el videojuego.}

\section{Descripción general de la empresa}
\paragraph{El presente videojuego, Genshin Impact, cuenta con una base de datos y servidores bastante sólidos que dan paso a una experiencia totalmente satisfactoria de los usuarios, sin problemática alguna respecto a la jugabilidad u otros percances. Sin embargo, MiHoyo ha detectado un pequeño abandono de tiempo de juego por parte de cientos de usuarios. Se especula que es debido a la cantidad de tiempo que los jugadores deben o tienen que invertir para poder tener “mejores personajes” dentro del juego. Para esto, debemos comprender que el juego, justamente, requiere de cierto tiempo considerable para poder ser “mejor”, y esto es lo que frecuentemente desanima a esos usuarios no acostumbrados a los videojuegos del estilo RPG. Y con la finalidad de recuperar a esos usuarios y atraer a muchos otros que busquen adentrarse en un nuevo mundo de posibilidades, MiHoyo propone el reto anteriormente mencionado. Cabe resaltar que el objetivo lanzado por el estudio no es facilitar en su totalidad el videojuego, puesto que pierde su “esencia” arduamente trabajada, si no que; aparte de buscar ese factor diferencial respecto a otros videojuegos, su principal objetivo sería guiar, ayudar y facilitar, hasta cierto límite, el videojuego y sus diversos factores que influyen en el desarrollo.
}

\section{Necesidad/usos de la base de datos}
\paragraph{La empresa Mihoyo está organizando una actualización donde se busca optimizar/añadir una ayuda a los usuarios, es por ello que hacer una buena implementación nos producirá que el trabajo sea lo mejor eficientemente posible para obtener diversas consultas en el menor tiempo posible. Esto permitirá una mejor interacción con el cliente, ayudará a la misma empresa para poder mejorar en el desarrollo del juego, también captar una mayor cantidad de jugadores y a su vez más ganancias.}

\section{¿Cómo resuelve el problema hoy?}

\subsection{¿Cómo se almacenan/procesan los datos hoy?}
\paragraph{Según lo investigado respecto a Genshin Impact, se sabe que los datos almacenados se refiere a la totalidad de objetos interactuables u observados en el mapa de los millones de usuarios. Además, se conoce que los personajes jugables tambien son almacenas junto con sus habilidades y demás data importante respecto a ello.}
\subsection{Flujo de datos}
\paragraph{La cuenta que cse rea puede ser linkeada con una o varias redes sociales, además de que esta permite acceder a la comunidad multijugador y guardar progresos dentro del videojuego.}
\paragraph{Además, el videojuego tiene una historia principal en la cual se recolectan objetos acorde a lo que ofrecen, además de misiones secundarias y eventos semanales los cuales sirven para que el usuario almacene todo lo obtenido en un inventario sincronizado con la cuenta.}

\section{Descripción detallada del sistema}
\subsection{Objetos de información actuales}
\paragraph{Para esta primera entrega tenemos como objetos de información la data de los jugadores de Genshin Impact y también el modelo entidad relación obtenida a partir de esta última.}

\subsection{Características y funcionalidades esperadas}
\paragraph{Se espera que esta base de datos pueda ser lo mejor eficientemente posible para dar una buena experiencia al usuario y pueda obtener información sobre sus personajes, como mejorarlos en base a la información de los otros usuarios y de esta manera darle una mejor experiencia a jugadores nuevos.}

\subsection{Tipos de usuarios existentes/necesarios}
\paragraph{El proyecto al estar referido a un videojuego, tanto offline como online (con un máximo de 4 jugadores), los únicos usuarios necesarios que deben existir para el desarrollo del videojuego son de aquellos que hayan iniciado el videojuego y por lo tanto posean un mapa de su propio mundo.}

\subsection{Tipos de consulta, actualizaciones}
\paragraph{Las consultas dentro del videojuego estarían enfocadas principalmente en base a los usuarios y que tanto se ha desarrollado dentro de su propio mundo abierto, donde se daría a conocer cientos de detalles que poseen todos los jugadores. Por otro lado, las actualizaciones se ven reflejadas en momentos donde el usuario adquiere o recoge objetos o incrementa factores respecto a sus personajes, y esto generaría una actualización respecto al cambio generado en cierto aspecto puesto que no hay necesidad de actualizar toda la información.}
\paragraph{Además, otro tipo de consultas serían dirigidas a estadísticas promedio de un jugador, por personaje adquirido o el poder total de todo su equipo, asimismo permitir actualizar los datos de los usuarios.}

\subsection{Tamaño estimado de la base de datos}
\begin{itemize}
    \item[1.- ] La base de datos se encuentra en constante expansión, ya que el videojuego, Genshin Impact, puede llegar a recibir cientos de jugadores que crean una cuenta por día.
    \item[2.- ]  Así mismo, continuamente y cada cierto tiempo, el equipo desarrollador agrega nuevas actualizaciones con nuevas innovadoras características de jugabilidad y contenido.
    \item[3.- ] En primer lugar, la tabla que tendrá una mayor actividad será la de “Objetos”, la cual sufre una alta cantidad de alteraciones puesto que un usuario puede tener más de 1000 objetos dentro de su cuenta. Posteriormente, la tabla de usuarios, junto con las tablas correspondientes a ciertos “eventos” también serán una de las más activas dentro del modelo.
    \item De forma general, la base de datos contará con nuevos registros para, prácticamente, toda relación dentro del modelo.
\end{itemize}

\section{ Objetivos del proyecto}
\paragraph{Manejar la verificación de de posesión de objetos y principalmente recolectar data que 
ayuden a tomar decisiones para mejorar el juego.}

\section{Referencias del proyecto}
\paragraph{El principal autor para el propio desarrollo del proyecto es el estudio MiHoyo. La data puede ser extraída únicamente del propio videojuego, Genshin Impact; y como de páginas web que previamente han recolectado la data mostrada dentro del videojuego. }

\section{Eventualidades}
\subsection{Problemas que pudieran encontrarse en el proyecto}
\paragraph{En primer lugar, la gran cantidad de entidades dentro del juego, dificultaba el desarrollo del modelo entidad-relación. Además, que dentro de cada item son diversos y se basan en la aleatoriedad las propiedades de algunas entidades, por lo que la definición de los atributos también mostraba un obstáculo en el desarrollo de entidades.}

\subsection{Límites y alcance del proyecto}
\paragraph{La cantidad gigantesca de jugadores junto con su respectiva data requieren de servidores completos y mucho más desarrollados comparados a los que se utilizarán de forma experimental.}

\chapter{Modelo Entidad Relación}
\section{Reglas semánticas}
\paragraph{Con la finalidad de implementar el modelo esperado debemos considerar las siguientes reglas semánticas que se observan en el videojuego.}

\begin{itemize}
    \item Debemos reconocer que existe un usuario con UID único, el cual maneja el resto de características del propio juego, desde el mapa hasta los objetos que contiene.
    \item El mapa del propio juego dependerá del usuario, donde el jugador puede desarrollarlo a su antojo. Además, notamos que el mapa está dividido en regiones, y esas regiones contienen cientos de NPC (Non-player character) que las habitan. Donde dichos NPC’s están clasificados como NPC’s que ofrecen objetos a cambio de otros, y NPC’s que solo contienen un diálogo interactivo con el jugador.
    \item En el mapa único de cada jugador se pueden realizar múltiples acciones limitadas, desde recoger hasta eliminar enemigos. Tales acciones conllevan una recompensa para el jugador de recolectables u objetos.
    \item Se sabe que los usuarios pueden armar un equipo donde los personajes predeterminados pueden estar incluidos, siendo un equipo formado por más de uno y menor o igual a 4 personajes jugables.
    \item Dichos personajes, de forma individual, poseen item’s clasificados en artefactos y armas, donde se puede equipar con 0 o 5 artefactos como máximo y necesariamente 1 arma. Además tanto los talentos como los atributos, dependen de un propio personaje jugable.
    \item Por último, el jugador puede acceder desde su menú a un apartado de eventos que se crean y tienen un tiempo límite para realizar las misiones asignadas; además de la tienda donde pueden adquirir ciertos objetos por un intercambio; también cuentan con un sistema gachapon\endnotetext{Son una variedad de cápsulas de juguetes dispensados en máquinas expendedoras y con mucha popularidad en Japón. "Gashapon"\ es una onomatopeya de dos sonidos: "gasha" (o "gacha") para la acción de arranque manual de una máquina expendedora de cápsulas de juguetes, y "pon"\  para el aterrizaje de la cápsula de juguete en la bandeja de recolección. "Gashapon"\ se usa tanto para las máquinas como para los juguetes que se obtienen de ellas.} donde por medio de otorgar protegidas puedes lanzar a un gachapon y obtener objetos más únicos.
\end{itemize}

\section{Modelo Entidad-Relación}
Revisar primer anexo \pageref{Entidad_Relacion}.

\section{Especificaciones y consideraciones sobre el proyecto}
\paragraph{En esta sección se describirá el modelo relacional para explicarlo.}
\begin{itemize}
    \item[$\blacksquare$] \textbf{Cuenta\_usuario}
        \begin{itemize}
            \item \textbf {UID: } El UserID es la clave númerica que va a identificar a cada jugador
            \item \textbf{Correo\_electrónico:} Todos los jugadores son identificados con un solo correo electrónico.
            \item \textbf{Username: } Es el nombre virtual que el jugador elige para el videojuego
            \item \textbf{Contraseña: }Sera el string con el que el usuario ingresa a su cuenta.
            \item \textbf{Cantidad\_resina: } El material denominado "Resina" pertenece al juego y es utilizado para reclamar recompensas mediante raids\endnote{Del inglés raid (incursión, redada).
            En juegos multijugador masivos (MMO), es un tipo de misión donde un gran número de jugadores, mucho mayor que el estándar del juego, se enfrenta a una misión o a un jefe enemigo de gran dificultad.\\
            Generalmente este tipo de misiones se instancian.}.
            
            \item \textbf{Protogemas: } Es un objeto importante en el juego, puesto que con las protogemas puedes adquirir cierta cantidad de "deseos". Los deseos son utilizados para adquirir personajes y armas por medio de los "Gachapones".
            \item \textbf{Cristales\_genesis: } Los cristales genesis se pueden adquirir mediante cierto pago real, y estos son utilizados para intercambiar por protogemas y por ende obtener mayor cantidad de oportunidades para los "Gachapones".
            \item \textbf{Cantidad\_mora: } El objeto "mora" son las monedas principales del videojuego, utilizadas para comprar materiales en tiendas dentro del juego.
        \end{itemize}
    \item[$\blacksquare$] \textbf{Objeto}
        \begin{itemize}
            \item \textbf{Id: } Es el Id que corresponden a los multiples objetos dentro del juego
            \item \textbf{Nombre: } Es el nombre que tiene cada objeto en el juego.
            \item \textbf{Tipo\_obj: } Es el tipo de objeto que se almacena, desde armas hasta materiales.
            \item \textbf{Obj\_estrellas: } Todos los objetos poseen una cantidad de estrellas, desde 1 hasta 5, que determinan su rareza dentro del juego. 
            \item \textbf{Cantidad: } Es la cantidad que los jugadores poseen de cada objeto.
        \end{itemize}
    \item [$\blacksquare$]\textbf{Evento\_Misiones}
        \begin{itemize}
            \item \textbf{Nombre: } Es el nombre que se le asigna al evento para las misiones.
            \item \textbf{Desde: } Fecha de inicio del evento.
            \item \textbf{Hasta: } Fecha de finalización del evento.
            \item \textbf {Completado: } Determina si el usuario completo cierto evento o no.
        \end{itemize}
    \item[$\blacksquare$]\textbf{Tienda}
        \begin{itemize}
            \item \textbf{Tipo\_obj: } Es el tipo de objeto que se adquiere en la tienda, desde personajes de baja calidad hasta materiales.
            \item \textbf{Cargo: } Es el monto o cantidad de moras que cuesta cada objeto.
            \item \textbf{Id: } Es el id del objeto que se esta adquiriendo.
            \item \textbf{Cantidad: } Es la cantidad de objetos que se estan comprando, mientras más objetos mayor será el cargo.
            \item \textbf{Nombre\_obj: } Es el nombre del objeto que estas adquiriendo.
            \item \textbf{Fecha: } Es la fecha en la que el jugador adquiere cierto objeto.
            \item \textbf{Código: } Refiere a un código de compra único que es creado al momento de adquirir un objeto.
            \item \textbf{Tipo\_divisa:} Es la moneda que se puede utilizar según el tipo en el apartado de la tienda.
        \end{itemize}
    \item[$\blacksquare$]\textbf{Gachapon}
        \begin{itemize}
            \item \textbf{Tipo: } Es el tipo de objeto que se adquiere al momento de "comprar" un deseo. 
            \item \textbf{Nombre\_obj: } Es el nombre del objeto adquirido con la compra de gachapones.
            \item \textbf{Fecha\_hora:} Es la fecha de adquisición del objeto comprado mediante el gachapon.
            \item \textbf{Código: } Es el código que es creado para identificar una compra y tirada en los gachapones
        \end{itemize}
    \item[$\blacksquare$]\textbf{Mapa}
        \begin{itemize}
            \item \textbf{Id\_mapa:} Este mapa es único para cada usuario y tiene sus propias criaturas,recolectables y mazmorras.
            \item \textbf{Porcentaje\_cmplt:} Es el porcentaje completado en las distintas regiones del mapa.
        \end{itemize}
    \item[$\blacksquare$]\textbf{Region}
        \begin{itemize}
            \item \textbf{Nombre:} Nombre de la región dentro del mapa del juego.
            \item \textbf{Porcentaje\_completado:} Es el porcentaje de misiones completados dentro de la región.
        \end{itemize}
    \item[$\blacksquare$]\textbf{Recompensa\_explo}
        \begin{itemize}
            \item \textbf{Posicion\_mapa: } Posición precisa donde se encuentra el evento a realizar.
            \item \textbf{Tipo\_acción:} Definiciones como eliminar criatura, recolectar, talar objetos para obtener materiales, entre otras.
        \end{itemize}
\end{itemize}

\chapter{Modelo relacional}
\section{Modelo relacional}

\begin{itemize}
    \item[$\rightarrow$]\onehalfspacing{ \justify{\textbf{Cuenta\_usuario} (\underline{UID:BIGINT}, username:VARCHAR(50),\\ correo\_electronico:VARCHAR(150), contraseña:VARCHAR(50), protogemas:BIGINT, cantidad\_resina:SMALLINT, cristales\_genesis:BIGINT, cantidad\_mora:BIGINT)}}

    \item[$\rightarrow$]\onehalfspacing{ \justify{\textbf{Evento\_Mision} (\underline{id:INT},Nombre:VARCHAR(100), desde:DATE, hasta:DATE)}}
    
    \item[$\rightarrow$]\onehalfspacing{ \justify{\textbf{Objeto} (\underline{Cuenta\_Usuario.UID:BIGINT}, \underline{id:INT},  Nombre:VARCHAR(100), cantidad:BIGINT ,tipo\_objeto:VARCHAR(30),objeto\_estrellas:VARCHAR(10)\\,tipo\_obj:VARCHAR(30))}}

    \item[$\rightarrow$]\onehalfspacing{ \justify{\textbf{Compra} (\underline{Codigo:BIGINT}, \underline{Cuenta\_Usuario.UID:BIGINT},\\ tipo\_div:VARCHAR(50), fecha:DATE, obj\_id:BIGINT, cantidad:SMALLINT), cargo:INT)}}

    \item[$\rightarrow$]\hyp{\onehalfspacing{\textbf{Gachapon} (\underline{fecha\_hora:TIME}, nombre:VARCHAR(50),tipo:VARCHAR(50), \underline{CMH.UID:BIGINT})}}
    
    \item[$\rightarrow$]\hyp{\onehalfspacing{ \textbf{Recompensa\_explo} (\underline{Cuenta\_Usuario.UID:BIGINT},\\ \underline{posicion\_mapa:VARCHAR(50)}, \underline{tipo\_accion:VARCHAR(50))}}}
    
    \item[$\rightarrow$]\hyp{\onehalfspacing{ \textbf{Mapa} (\underline{Cuenta\_Usuario..UID:BIGINT}, \underline{id\_mapa:BIGINT}, porcentaje\_completo\_m:DOUBLEPRESICION)}}

    \item[$\rightarrow$]\hyp{\onehalfspacing{\textbf{Region} (\underline{Cuenta\_Usuario.UID:BIGINT}, \underline{Id\_mapa:BIGINT}, \\ \underline{nombre\_r:VARCHAR(255)}, porcentaje\_completado\_r:DOUBLEPRESICION, cant\_tp\_act:BIGINT, cant\_boss\_act:BIGINT, \\ cant\_dominios\_act:BIGINT,cant\_estatuas\_act:BIGINT)
    }}

\end{itemize}

\section{Especificaciones de transformación}
\subsection{Entidades}

\begin{center}    
    \begin{tabular}{|p{3cm}|p{5cm}|}
    \hline
    \multicolumn{2}{|c|}{Cuenta\_usuario} \\ 
    \hline
    \hline
    Primary Key& UID\\ 
    \hline
    Foreign Key & -\\ 
    \hline
    \end{tabular}
\end{center}

\begin{center}    
    \begin{tabular}{|p{3cm}|p{5cm}|}
    \hline
    \multicolumn{2}{|c|}{Objeto} \\ 
    \hline
    \hline
    Primary Key& Cuenta\_Usuario.UID\\ 
    \hline
    Foreign Key& Cuenta\_Usuario.UID\\ 
    \hline
    \end{tabular}
\end{center}

\begin{center}    
    \begin{tabular}{|p{3cm}|p{5cm}|}
    \hline
    \multicolumn{2}{|c|}{Mapa} \\ 
    \hline
    \hline
    Primary Key & id\_mapa, Cuenta\_Usuario.UID\\
    \hline
    Foreign Key& Cuenta\_Usuario.UID\\ 
    \hline
    \end{tabular}
\end{center}

\begin{center}    
    \begin{tabular}{|p{3cm}|p{5cm}|}
    \hline
    \multicolumn{2}{|c|}{Region} \\ 
    \hline
    \hline
    Primary Key& -\\ 
    \hline
    Foreign Key& nombre\\ 
    \hline
    \end{tabular}
\end{center}
\begin{center}    
    \begin{tabular}{|p{3cm}|p{5cm}|}
    \hline
    \multicolumn{2}{|c|}{Evento\_Misiones} \\ 
    \hline
    \hline
    Primary Key& desde,hasta,nombre\\ 
    \hline
    Foreign Key &-\\ 
    \hline
    \end{tabular}
\end{center}

\begin{center}    
    \begin{tabular}{|p{3cm}|p{5cm}|}
    \hline
    \multicolumn{2}{|c|}{Tienda} \\ 
    \hline
    \hline
    Primary Key& código\\ 
    \hline
    Foreign Key &-\\ 
    \hline
    \end{tabular}
\end{center}

\begin{center}    
    \begin{tabular}{|p{3cm}|p{5cm}|}
    \hline
    \multicolumn{2}{|c|}{Gachapon} \\ 
    \hline
    \hline
    Primary key&código\\
    \hline
    Foreign Key &-\\ 
    \hline
    \end{tabular}
\end{center}

\begin{center}    
    \begin{tabular}{|p{3cm}|p{5cm}|}
    \hline
    \multicolumn{2}{|c|}{Recompensa\_explo} \\ 
    \hline
    \hline
    Primary key&posición\_mapa,tipo\_acción\\
    \hline
    Foreign Key &-\\ 
    \hline
    \end{tabular}
\end{center}

\subsection{Relaciones binarias}

\begin{center}    
    \begin{tabular}{|p{3cm}|p{5cm}|}
    \hline
    Nombre de relación& otorga\_em\\ 
    \hline
    \hline
    \multirow{2}{6em}{Entidades relacionadas}&Evento\_misiones \\ &Objeto\\ 
    \hline
    \end{tabular}
\end{center}

\begin{center}    
    \begin{tabular}{|p{3cm}|p{5cm}|}
    \hline
    Nombre de relación& otorga\_t\\ 
    \hline
    \hline
    \multirow{2}{6em}{Entidades relacionadas}& Tienda \\ &Objeto\\ 
    \hline
    \end{tabular}
\end{center}

\begin{center}    
    \begin{tabular}{|p{3cm}|p{5cm}|}
    \hline
    Nombre de relación& otorga\_g\\ 
    \hline
    \hline
    \multirow{2}{6em}{Entidades relacionadas}& Gachapon \\ &Objeto\\ 
    \hline
    \end{tabular}
\end{center}

\begin{center}    
    \begin{tabular}{|p{3cm}|p{5cm}|}
    \hline
    Nombre de relación& Crea\\ 
    \hline
    \hline
    \multirow{2}{6em}{Entidades relacionadas}&Equipo \\ &CuentaMihoyo/Usuario\\ 
    \hline
    \end{tabular}
\end{center}

\begin{center}    
    \begin{tabular}{|p{3cm}|p{5cm}|}
    \hline
    Nombre de relación& Juega\_en\\ 
    \hline
    \hline
    \multirow{2}{6em}{Entidades relacionadas}& Mapa \\ &Cuenta\_usuario\\ 
    \hline
    \end{tabular}
\end{center}

\begin{center}    
    \begin{tabular}{|p{3cm}|p{5cm}|}
    \hline
    Nombre de relación& contiene\\ 
    \hline
    \hline
    \multirow{2}{6em}{Entidades relacionadas}& Mapa \\ &Región\\ 
    \hline
    \end{tabular}
\end{center}

\begin{center}    
    \begin{tabular}{|p{3cm}|p{5cm}|}
    \hline
    Nombre de relación& Realizan\\ 
    \hline
    \hline
    \multirow{2}{6em}{Entidades relacionadas}& Personajes \\ &Evento\_acción\\ 
    \hline
    \end{tabular}
\end{center}


\begin{center}    
    \begin{tabular}{|p{3cm}|p{5cm}|}
    \hline
    Nombre de relación& otorga\_ea\\ 
    \hline
    \hline
    \multirow{2}{8em}{Entidades\\ relacionadas}& objeto \\ &Evento\_acción \\ 
    \hline
    \end{tabular}
\end{center}

%TODO: Cambiar imagenes a tablas

\subsection{Diccionario de datos}
\includegraphics[scale = 0.7]{1_p.jpg}
\newpage
\includegraphics[scale = 0.8]{3_p.jpg}
\includegraphics[scale = 0.7]{4_p.jpg}
\newpage
\includegraphics[scale = 0.7]{5_p.jpg}
\includegraphics[scale = 0.7]{6_p.jpg}
\newpage
\includegraphics[scale = 0.7]{7_p.jpg}
\includegraphics[scale = 0.7]{8_p.jpg}
\newpage
\includegraphics[scale = 0.7]{9_p.jpg}
\includegraphics[scale = 0.7]{10_p.jpg}
\newpage
\includegraphics[scale = 0.7]{11_p.jpg}
\includegraphics[scale = 0.7]{12_p.jpg}


\chapter{Significados} 
\theendnotes
\appendix
\chapter{Modelo Entidad-relación} \label{Entidad_Relacion}
\includegraphics[scale = 0.50, angle=90]{AnexoA_1.jpg}
\newpage
\includegraphics[scale = 0.50, angle=90]{AnexoA_2.jpg}
\newline
\newline
Para apreciar mejor el modelo, se añade link a Miro con el modelo \href{https://miro.com/welcomeonboard/cm95QTlodklVRjI5cEtYaUF5UmprejR0OEpEazk3dzRmVkYxZ2hrVFdmVmc2RDl5NVpjeTdwenRYQThsSm9VdHwzMDc0NDU3MzQ4NTY5OTQ5NzE0}{'Entidad-Relación' y notas}




\end{document}
